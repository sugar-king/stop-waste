\chapter{Implementacija i korisničko sučelje}
		
		
		\section{Korištene tehnologije i alati}
		
			\textbf{\textit{}}
			
Komunikacija unutar tima:
\vspace{4mm}
	
WhatsApp

Komunikacija unutar tima ostvarena je korištenjem aplikacije WhatsApp. 
WhatsApp je aplikacija za komunikaciju korisnika. Omogućava komunikaciju u stvarnom vremenu putem tekstualnih poruka, glasovnih poruka ili videopoziva. Podržava mogućnost dijeljenja slika, dokumenata i videa.

Službena stranica: https://www.whatsapp.com/
\vspace{3mm}

Microsoft Teams

Video sastanci tima, zajedničko uređivanje i diejljenje dokumenata održavani su putem aplikacije Microsoft Teams. 
Microsoft Teams je aplikacija koja nudi radni prostor timovima u obliku videokonferencija, poruka, pohrane datoteka i zajedničkog uređivanja dokumenata u stvarnom vremenu. 

Službena stranica: https://www.microsoft.com/hr-hr/microsoft-365/microsoft-teams/group-chat-software

\vspace{8mm}
Dokumentacija:
\vspace{4mm}

Astah UML 

Za izradu svih UML dijagrama korišten je Astah UML.
Astah UML je alat za stvaranje UML dijagrama. Omogućava stvaranje  dijagrama obrazaca uporabe, dijagrama razreda, dijagrama stanja, dijagrama aktivnosti, sekvencijskih dijagrama, dijagrama komponenata, dijagrama komunikacije, složene strukture, razmještaja i umnih mapa. 

Službena stranica: https://astah.net
\vspace{3mm}

Texmaker

Za pisanje dokumentacije koristili smo Texmaker.
Texmaker je besplatni višeplatformski LaTeX uređivač. Uključuje mnoge alate potrebne za razvoj dokumenata s LaTeX-om, podršku za unicode, provjeru pravopisa, automatsko dovršavanje i ugrađeni pdf preglednik.

Službena stranica: https://www.xm1math.net/texmaker/
\vspace{8mm}

Rad na aplikaciji:
\vspace{4mm}

Git i GitLab

Git je korišten kao sustav za upravljanje izvornim kodom, dok je udaljeni reozitorij projekta dostupan na web platformi GitLab.
Git je sustav otvorenog koda namijenjen upravljanju izvornim kodom. Pruža podršku za razvoj, stvaranje i spajanje grana.
GitLab je web platforma koja pruža upravljanje Git repozirotijem i kontinuiranu integraciju uz praćenje verzija. 

Službene stranice: https://git-scm.com/

	https://gitlab.com/
\vspace{3mm}

H2

Kao početna baza podataka koristio se H2 sustav.
H2 je sustav za upravljanje relacijskom bazom podataka napisanom u programskom jeziku Java. Omogućava čuvanje svih podataka u memoriji i podaci se osvježavaju pri svakom pokretanju aplikacije.

Službena stranica:https://www.h2database.com/html/main.html 
\vspace{3mm}

PostgreSQL i pgAdmin

Stalna baza u aplikaciji izrađena je pomoću PostgreSQL u pgAdminu.
PostgreSQL je bespaltan sustav otvorenog koda za upravljanje relacijskom bazom podataka. Podržava SQL upite i provjerava sigurnost.
PgAdmin je bespaltni administratorki alat s grafičkim korisničkim sučeljem za upravljanje PostgreSQL-om.

Službena stranica: https://www.pgadmin.org/
\vspace{3mm}

Intellij IDEA

Za pisanje koda korišten je Intellij IDEA
Intellij IDEA je integrirana razvojna okolina za razvoj programske potpore. Sadrži integrirani sustav za upravljanje inačicama. Najpopularnija razvojna okolina za pisanje programske potpore u programskom jeziku Java.

Službena stranica: https://www.jetbrains.com/idea/
\vspace{3mm}

HTML

Za prikaz web aplikacije koristi se HTML. 
HTML je standardni označni jezik i u njemu se pišu dokumenti koji se prikazuju u web pregledniku. Opisuje semantičku strukturu web stranice.
\vspace{3mm}

CSS

Stilska prezentacija web stranice izvdena je pomoću CSS-a.
CSS je stilski jezik za opis prezentacije dokumenta napisanog u označnom jeziku i omogućava odvajanje sadržaja od prezentacije što omogućava veću fleksibilnost pri izradi. 

\vspace{8mm}
Za izradu fronteda koristili smo React i JavaScript. 
\vspace{4mm}

JavaScript

Dinamičko kreiranje web stranice izvedeno je pomoću JavaScript-a. JavaScript je skriptni programski jezik koji se izvršava u internet pregledniku
na strani korisnika.

Službena stranica: https://www.javascript.com/
\vspace{3mm}

React

React je biblioteka u JavaScriptu za izgradnju korisničkih sučelja. Održavana je od strane Facebooka. 
React se najčešće koristi kao osnova u razvoju web ili mobilnih aplikacija. 

Službena stranica: https://reactjs.org/
\vspace{3mm}

SpringBoot

Za izradu pozadinske aplikacije(engl. back-end) koristili smo SpringBoot. Pozadinska aplikacija pisana je u razvojnom okruženju IntelliJ, koje je
omogućilo korištenje okvira Spring Boot koji je baziran na programskom jeziku Java. Spring Boot je projekt koji se oslanja na Spring Framework i koji omogućuje puno učinkovitiji i brži pristup izgradnji Spring aplikacija.

Službena stranica: https://spring.io/projects/spring-boot
\vspace{5mm}

			
			
			\eject 
		
	
		\section{Ispitivanje programskog rješenja}
			
		Provedeno je ispitivanje programskog rješenja u svrhu pronalaska grešaka i nepredviđenog ponašanja sustava na korisnikove akcije. Testiranje se sastoji od da dijela, testiranja komponenti sustava te testiranje sustava u cjelini. U svrhu testiranja komponenti programskog rješenja korišten je alat JUnit. Ponašanje sustava je testirano pomoću alata Selenium WebDriver za Google Chrome
			
			\subsection{Ispitivanje komponenti}
			Ispitivanjem komponenti testirana je funkcionalnost razreda koji implementiraju temeljne funkcionalnosti sustava.
			
			\textbf{Ispitni slučaj 1: Testiranje dohvata podataka o korisniku iz baze podataka premu username-u}
			
			Testira se dohvaća li metoda findByUsername(String username) doista korisnika čije smo podatke zapisali u bazu podataka.
			
			
			
			\begin{figure}[H]
				\includegraphics[scale=0.7]{slike/test1.PNG} %veličina slike u odnosu na originalnu datoteku i pozicija slike
				\centering
				\caption{Testiranje dohvata korisnika po username-ui}
				\label{fig:test1}
			\end{figure}
			
			
			
			
			
				\textbf{Ispitni slučaj 2: Testiranje postoji li korisnik u bazi podataka po username-u}
				
			Testira se funkionalnost metode findByUsername(String username) tako da se pomoću metode existsByUsername(String username) provjerava postoji li korisnik u bazi podataka ako postoji da će
			metoda findByUsername(String username) također pronaći zapis u korisniku u bazi podataka, u suprotnom nijedna od navedenih metoda neće pronaći podatke o korisniku.
				
				
				\begin{figure}[H]
					\includegraphics[scale=0.7]{slike/test2.PNG} %veličina slike u odnosu na originalnu datoteku i pozicija slike
					\centering
					\caption{Testiranje postoji li korisnik u bazi podataka po username-u}
					\label{fig:test2}
				\end{figure}
			
			\textbf{Ispitni slučaj 3: Testiranje postoji li korisnik u bazi podataka po email-u}
			
		Testira se funkionalnost metode findByEmail(String email) tako da se pomoću metode existsByEmail(String email) provjerava postoji li korisnik u bazi podataka te ako postoji da će metoda findByEmail(String email) također pronaći zapis o korisniku u bazi podatak, u suprotnom nijedna od navedenih metoda neće pronaći podatke o korisniku.
			
			\begin{figure}[H]
				\includegraphics[scale=0.7]{slike/test3.PNG} %veličina slike u odnosu na originalnu datoteku i pozicija slike
				\centering
				\caption{Testiranje postoji li korisnik u bazi podataka po email-u}
				\label{fig:test3}
			\end{figure}
		
		U svrhu sljedećih ispitivanja koriste se dvojnici odnosno Mock objekti kojima možemo simulirati radi stavrnih objekata
		
		\begin{figure}[H]
			\includegraphics[scale=0.7]{slike/test3a.PNG} %veličina slike u odnosu na originalnu datoteku i pozicija slike
			\centering
			\caption{Mock objekti}
			\label{fig:test3a}
		\end{figure}
		
		
		\textbf{Ispitni slučaj 4: Testiranje registracije}
		Testiramo dobivamo li za povratnu vrijednost metode za registraciju korisnika doista korisnika kojeg smo htjeli registrirati.
		
		
		\begin{figure}[H]
			\includegraphics[scale=0.7]{slike/test4.PNG} %veličina slike u odnosu na originalnu datoteku i pozicija slike
			\centering
			\caption{Testiranje registracije korisnika}
			\label{fig:test4}
		\end{figure}
	
	\textbf{Ispitni slučaj 5 : Testiranje jedinstvenosti email-a}
	
	Testiramo mogu li se registrirati dva korisnika s identičnom email adresom, očekivano ponašanje je bacanje iznimke RequestDeniedException. 
	
	\begin{figure}[H]
		\includegraphics[scale=0.7]{slike/test5.PNG} %veličina slike u odnosu na originalnu datoteku i pozicija slike
		\centering
		\caption{Testiranje jedinstvenosti email-a}
		\label{fig:test5}
	\end{figure}
			
			
				\textbf{Ispitni slučaj 6 : Testiranje jedinstvenosti username-a}
			
		Testiramo mogu li se registrirati dva korisnika s identičnim korisničkim imenom, očekivano ponašanje je bacanje iznimke RequestDeniedException. 
			
			\begin{figure}[H]
				\includegraphics[scale=0.7]{slike/test5.PNG} %veličina slike u odnosu na originalnu datoteku i pozicija slike
				\centering
				\caption{Testiranje jedinstvenosti username-a}
				\label{fig:test6}
			\end{figure} 
		
			\subsection{Ispitivanje sustava}
			
			Ispitivanje sustava je provedeno pomoću alata Selenium WebDriver za web preglednik Google Chrome. Testirani su slučajevi registracije i prijave korisnika te objava oglasa, rezervacija oglasa i slanje poruka.
			
			
			\textbf{Ispitni slučaj 1: Ispitivanje registrcije}
			
		Pomoću konfiguriranog Chreome drivrera dohvaća se stranica za registraciju i šalju se potrebni podatci za registraciju korisnika, prvim izvođenjem testa email korisnika i korisničko ime su jedinstveni te je očekivani rezultat uspješna registracija.
			
			
			
			\begin{figure}[H]
				\includegraphics[scale=0.9]{slike/sel1.PNG} %veličina slike u odnosu na originalnu datoteku i pozicija slike
				\centering
				\caption{Testiranje registracije}
				\label{fig:sel1}
			\end{figure}
			
			
			
			
			
			\textbf{Ispitni slučaj 2: Testiranje prijave postojećeg korisnika
				}
			
			Testiramo prijavu u aplikaciju pomoću unaprijed registriranog korisnika, očekivano ponašanje je uspješna prijava što dokazujemo ispisivanje korisničkog imena prijavljenog korisnika u gornjem desnom kutu stranice.
			
			
			\begin{figure}[H]
				\includegraphics[scale=0.9]{slike/sel2.PNG} %veličina slike u odnosu na originalnu datoteku i pozicija slike
				\centering
				\caption{Testiranje prijave postojećeg korisnika}
				\label{fig:sel2}
			\end{figure}
			
			\textbf{Ispitni slučaj 3: Testiranje prijave nepostojećeg korisnika
				}
			
			Testiramo možemo li se prijaviti u aplikaciju s nepostojećim korisničkim imenom, očekivano ponašanje je odbijanje prijave u sustav te ispis prigodne poruke korisniku.
			
			\begin{figure}[H]
				\includegraphics[scale=0.9]{slike/sel3.PNG} %veličina slike u odnosu na originalnu datoteku i pozicija slike
				\centering
				\caption{Testiranje prijave nepostojećeg korisnika}
				\label{fig:sel3}
			\end{figure}
			
			
			
			
			\textbf{Ispitni slučaj 4: Testiranje slanja poruka
				}
			Testiramo slanje poruka između dvaju postojećih korisnika. Prijavljujemo se na jedan korisnički račun (metoda login(String username, String password)), zatim šaljemo poruku korisiniku i odjavljujemo se kako bismo se mogli prijaviti na drugi korisnički račun. Nakon prijave korisnika koji je trebao zaprimiti poruku i provjeraamo postoji li poruka koju smo poslali.
			
			
			\begin{figure}[H]
				\includegraphics[scale=0.9]{slike/sel4.PNG} %veličina slike u odnosu na originalnu datoteku i pozicija slike
				\centering
				\caption{Testiranje slanja poruke}
				\label{fig:sel4}
			\end{figure}
			
			\textbf{Ispitni slučaj 5: Testiranje objave oglasa
				}
			
			Testiramo objavu oglasa tako s početne stranice te nakon toga provjeravamo je li se upravo objavljeni oglas pojavio u korisnikovoj listi njegovih objavljenih oglasa. 
			
			\begin{figure}[H]
				\includegraphics[scale=0.9]{slike/sel5.PNG} %veličina slike u odnosu na originalnu datoteku i pozicija slike
				\centering
				\caption{Testiranje objave oglasa}
				\label{fig:sel5}
			\end{figure}
			
			\eject 
		
		\section{Upute za puštanje u pogon}
		
			\textbf{\textit{Instalacija potrebnih aplikacija}}
			
			Za pokretanje ove aplikacije kao poslužitelja potrebno je imati instalirano nekoliko aplikacija koje omogućavaju pokretanje i sam rad aplikacije.
		
			Za bazu je potrebno instalirati pgAdmin koji omogućava povezivanje na PostgreSQL bazu podataka.Poveznica za preuzimanje je: \url{https://www.pgadmin.org/download/} ali se i u  popisu literature nalazi poveznica na kojemu je moguće preuzeti pgAdmin te postoje detaljne upute kako se može postaviti sve potrebno za rad baze.
		
			Na računalu je svakako potrebno imati podršku za prevođenje i pokretanje aplikacija napisanih u programskom jeziku Java. Također poveznica za preuzimanje potrebne podrške: \url{http://jdk.java.net/11/} te je dodana u popisu literature te se lako mogu pronaći upute koje će vas voditi kroz postavljanje svega potrebnog za uspostavu podrške za rad s programskim jezikom Java. Nakon instalacije je potrebno provjeriti radi li sve uspješno, a to je najjednostavnije provjeriti u naredbenom retku unosom naredbe "java -version". U danom primjeru je instalirana verzija 15, no za izvođenje programa je sasvim dovoljna verzija 11. 
			
			\begin{figure}[H]
				\includegraphics[scale=0.6]{slike/Java.PNG} %veličina slike u odnosu na originalnu datoteku i pozicija slike
				\centering
				\caption{Provjera instalacije Jave}
				\label{fig:java}
			\end{figure}
			
			Kako se uz Javu koristi i programski jezik JavaScript potrebno je instalirati i njegovu podršku odnosno potrebno je instalirati Node.js. Poveznica za instalaciju Node.js-a: \url{https://nodejs.org/en/download/} te je dana u popisu literature te se jednostavno uz pomoć uputa instalira. Također je potrebno provjeriti uspješnost instaliranja Node.js-a u naredbenom retku naredbom: " node -v".
			
			\begin{figure}[H]
				\includegraphics[scale=0.6]{slike/node.PNG} %veličina slike u odnosu na originalnu datoteku i pozicija slike
				\centering
				\caption{Provjera instalacije Node.js-a}
				\label{fig:node}
			\end{figure}
		
		Kada je sve to spremno potrebno je instalirati razvojnu okolinu koja nam pomaže pri izradi same aplikacije i njenog pokretanja. Primjer jedne takve okoline je Intelij IDE koji je za studente FER-a besplatan, no moguće je koristiti i druge razvojne okoline poput Eclipsa. Poveznica za preuzimanje ove razvojne okoline:  \url{https://www.jetbrains.com/idea/download/#section=windows}  također se nalazi i u popisu literature.
		
		Uz to za preuzimanje projekta možete se poslužiti sa Git GUI-em kojega je moguće preuzeti sa slijedeće poveznice: \url{https://git-scm.com/downloads}. Međutim možete samo skinuti projekt sa \url{https://gitlab.com/ivicamarica/ivicamarica} te ga pohraniti na željeno mjesto.
		
		Za kraj morate na svome računalu imati i program za upravljanje projektima, u ovom slučaju je to Gradle. \url{https://gradle.org/install/} je poveznica za prezimanje.
		
		Kada ste sve to pripremili konačno možete pokrenuti aplikaciju. Za uspješno pokretanje aplikacije potrebno je prvo instalirati npm pakete koji se koriste. To se treba učiniti pokretanjem naredbe
		npm install unutar direktorija src/main/js.
		Spring Boot aplikaciju je moguće pokrenuti unutar IDE-a ili naredbom gradle bootRun iz naredbenog retka.
		Frontend se pokreće naredbom npm start unutar direktorija src/main/js nakon čega se aplikacija može pregledavati u pregledniku na adresi http://localhost:3000.
		Nakon što ste ovako pokrenuli aplikaciju možete ju koristiti jer će se ona sama pobrinuti za sve potrebne komponente poput stvaranja baze i ostaloga.
		
		Ukoliko samo želite pogledati i isprobati aplikaciju, aplikacija je dostupna na poveznici  \url{https://progi-stop-waste.herokuapp.com/}.
		
		
		
		
			 
			
			\eject 